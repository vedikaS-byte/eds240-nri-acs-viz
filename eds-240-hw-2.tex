% Options for packages loaded elsewhere
\PassOptionsToPackage{unicode}{hyperref}
\PassOptionsToPackage{hyphens}{url}
\PassOptionsToPackage{dvipsnames,svgnames,x11names}{xcolor}
%
\documentclass[
  letterpaper,
  DIV=11,
  numbers=noendperiod]{scrartcl}

\usepackage{amsmath,amssymb}
\usepackage{lmodern}
\usepackage{iftex}
\ifPDFTeX
  \usepackage[T1]{fontenc}
  \usepackage[utf8]{inputenc}
  \usepackage{textcomp} % provide euro and other symbols
\else % if luatex or xetex
  \usepackage{unicode-math}
  \defaultfontfeatures{Scale=MatchLowercase}
  \defaultfontfeatures[\rmfamily]{Ligatures=TeX,Scale=1}
\fi
% Use upquote if available, for straight quotes in verbatim environments
\IfFileExists{upquote.sty}{\usepackage{upquote}}{}
\IfFileExists{microtype.sty}{% use microtype if available
  \usepackage[]{microtype}
  \UseMicrotypeSet[protrusion]{basicmath} % disable protrusion for tt fonts
}{}
\makeatletter
\@ifundefined{KOMAClassName}{% if non-KOMA class
  \IfFileExists{parskip.sty}{%
    \usepackage{parskip}
  }{% else
    \setlength{\parindent}{0pt}
    \setlength{\parskip}{6pt plus 2pt minus 1pt}}
}{% if KOMA class
  \KOMAoptions{parskip=half}}
\makeatother
\usepackage{xcolor}
\setlength{\emergencystretch}{3em} % prevent overfull lines
\setcounter{secnumdepth}{-\maxdimen} % remove section numbering
% Make \paragraph and \subparagraph free-standing
\ifx\paragraph\undefined\else
  \let\oldparagraph\paragraph
  \renewcommand{\paragraph}[1]{\oldparagraph{#1}\mbox{}}
\fi
\ifx\subparagraph\undefined\else
  \let\oldsubparagraph\subparagraph
  \renewcommand{\subparagraph}[1]{\oldsubparagraph{#1}\mbox{}}
\fi

\usepackage{color}
\usepackage{fancyvrb}
\newcommand{\VerbBar}{|}
\newcommand{\VERB}{\Verb[commandchars=\\\{\}]}
\DefineVerbatimEnvironment{Highlighting}{Verbatim}{commandchars=\\\{\}}
% Add ',fontsize=\small' for more characters per line
\usepackage{framed}
\definecolor{shadecolor}{RGB}{241,243,245}
\newenvironment{Shaded}{\begin{snugshade}}{\end{snugshade}}
\newcommand{\AlertTok}[1]{\textcolor[rgb]{0.68,0.00,0.00}{#1}}
\newcommand{\AnnotationTok}[1]{\textcolor[rgb]{0.37,0.37,0.37}{#1}}
\newcommand{\AttributeTok}[1]{\textcolor[rgb]{0.40,0.45,0.13}{#1}}
\newcommand{\BaseNTok}[1]{\textcolor[rgb]{0.68,0.00,0.00}{#1}}
\newcommand{\BuiltInTok}[1]{\textcolor[rgb]{0.00,0.23,0.31}{#1}}
\newcommand{\CharTok}[1]{\textcolor[rgb]{0.13,0.47,0.30}{#1}}
\newcommand{\CommentTok}[1]{\textcolor[rgb]{0.37,0.37,0.37}{#1}}
\newcommand{\CommentVarTok}[1]{\textcolor[rgb]{0.37,0.37,0.37}{\textit{#1}}}
\newcommand{\ConstantTok}[1]{\textcolor[rgb]{0.56,0.35,0.01}{#1}}
\newcommand{\ControlFlowTok}[1]{\textcolor[rgb]{0.00,0.23,0.31}{#1}}
\newcommand{\DataTypeTok}[1]{\textcolor[rgb]{0.68,0.00,0.00}{#1}}
\newcommand{\DecValTok}[1]{\textcolor[rgb]{0.68,0.00,0.00}{#1}}
\newcommand{\DocumentationTok}[1]{\textcolor[rgb]{0.37,0.37,0.37}{\textit{#1}}}
\newcommand{\ErrorTok}[1]{\textcolor[rgb]{0.68,0.00,0.00}{#1}}
\newcommand{\ExtensionTok}[1]{\textcolor[rgb]{0.00,0.23,0.31}{#1}}
\newcommand{\FloatTok}[1]{\textcolor[rgb]{0.68,0.00,0.00}{#1}}
\newcommand{\FunctionTok}[1]{\textcolor[rgb]{0.28,0.35,0.67}{#1}}
\newcommand{\ImportTok}[1]{\textcolor[rgb]{0.00,0.46,0.62}{#1}}
\newcommand{\InformationTok}[1]{\textcolor[rgb]{0.37,0.37,0.37}{#1}}
\newcommand{\KeywordTok}[1]{\textcolor[rgb]{0.00,0.23,0.31}{#1}}
\newcommand{\NormalTok}[1]{\textcolor[rgb]{0.00,0.23,0.31}{#1}}
\newcommand{\OperatorTok}[1]{\textcolor[rgb]{0.37,0.37,0.37}{#1}}
\newcommand{\OtherTok}[1]{\textcolor[rgb]{0.00,0.23,0.31}{#1}}
\newcommand{\PreprocessorTok}[1]{\textcolor[rgb]{0.68,0.00,0.00}{#1}}
\newcommand{\RegionMarkerTok}[1]{\textcolor[rgb]{0.00,0.23,0.31}{#1}}
\newcommand{\SpecialCharTok}[1]{\textcolor[rgb]{0.37,0.37,0.37}{#1}}
\newcommand{\SpecialStringTok}[1]{\textcolor[rgb]{0.13,0.47,0.30}{#1}}
\newcommand{\StringTok}[1]{\textcolor[rgb]{0.13,0.47,0.30}{#1}}
\newcommand{\VariableTok}[1]{\textcolor[rgb]{0.07,0.07,0.07}{#1}}
\newcommand{\VerbatimStringTok}[1]{\textcolor[rgb]{0.13,0.47,0.30}{#1}}
\newcommand{\WarningTok}[1]{\textcolor[rgb]{0.37,0.37,0.37}{\textit{#1}}}

\providecommand{\tightlist}{%
  \setlength{\itemsep}{0pt}\setlength{\parskip}{0pt}}\usepackage{longtable,booktabs,array}
\usepackage{calc} % for calculating minipage widths
% Correct order of tables after \paragraph or \subparagraph
\usepackage{etoolbox}
\makeatletter
\patchcmd\longtable{\par}{\if@noskipsec\mbox{}\fi\par}{}{}
\makeatother
% Allow footnotes in longtable head/foot
\IfFileExists{footnotehyper.sty}{\usepackage{footnotehyper}}{\usepackage{footnote}}
\makesavenoteenv{longtable}
\usepackage{graphicx}
\makeatletter
\def\maxwidth{\ifdim\Gin@nat@width>\linewidth\linewidth\else\Gin@nat@width\fi}
\def\maxheight{\ifdim\Gin@nat@height>\textheight\textheight\else\Gin@nat@height\fi}
\makeatother
% Scale images if necessary, so that they will not overflow the page
% margins by default, and it is still possible to overwrite the defaults
% using explicit options in \includegraphics[width, height, ...]{}
\setkeys{Gin}{width=\maxwidth,height=\maxheight,keepaspectratio}
% Set default figure placement to htbp
\makeatletter
\def\fps@figure{htbp}
\makeatother

\KOMAoption{captions}{tableheading}
\makeatletter
\makeatother
\makeatletter
\makeatother
\makeatletter
\@ifpackageloaded{caption}{}{\usepackage{caption}}
\AtBeginDocument{%
\ifdefined\contentsname
  \renewcommand*\contentsname{Table of contents}
\else
  \newcommand\contentsname{Table of contents}
\fi
\ifdefined\listfigurename
  \renewcommand*\listfigurename{List of Figures}
\else
  \newcommand\listfigurename{List of Figures}
\fi
\ifdefined\listtablename
  \renewcommand*\listtablename{List of Tables}
\else
  \newcommand\listtablename{List of Tables}
\fi
\ifdefined\figurename
  \renewcommand*\figurename{Figure}
\else
  \newcommand\figurename{Figure}
\fi
\ifdefined\tablename
  \renewcommand*\tablename{Table}
\else
  \newcommand\tablename{Table}
\fi
}
\@ifpackageloaded{float}{}{\usepackage{float}}
\floatstyle{ruled}
\@ifundefined{c@chapter}{\newfloat{codelisting}{h}{lop}}{\newfloat{codelisting}{h}{lop}[chapter]}
\floatname{codelisting}{Listing}
\newcommand*\listoflistings{\listof{codelisting}{List of Listings}}
\makeatother
\makeatletter
\@ifpackageloaded{caption}{}{\usepackage{caption}}
\@ifpackageloaded{subcaption}{}{\usepackage{subcaption}}
\makeatother
\makeatletter
\@ifpackageloaded{tcolorbox}{}{\usepackage[many]{tcolorbox}}
\makeatother
\makeatletter
\@ifundefined{shadecolor}{\definecolor{shadecolor}{rgb}{.97, .97, .97}}
\makeatother
\makeatletter
\makeatother
\ifLuaTeX
  \usepackage{selnolig}  % disable illegal ligatures
\fi
\IfFileExists{bookmark.sty}{\usepackage{bookmark}}{\usepackage{hyperref}}
\IfFileExists{xurl.sty}{\usepackage{xurl}}{} % add URL line breaks if available
\urlstyle{same} % disable monospaced font for URLs
\hypersetup{
  pdftitle={HW \#2: Visualizing FEMA NRI Data},
  colorlinks=true,
  linkcolor={blue},
  filecolor={Maroon},
  citecolor={Blue},
  urlcolor={Blue},
  pdfcreator={LaTeX via pandoc}}

\title{HW \#2: Visualizing FEMA NRI Data}
\author{}
\date{}

\begin{document}
\maketitle
\ifdefined\Shaded\renewenvironment{Shaded}{\begin{tcolorbox}[interior hidden, enhanced, frame hidden, borderline west={3pt}{0pt}{shadecolor}, breakable, sharp corners, boxrule=0pt]}{\end{tcolorbox}}\fi

\hypertarget{hw-2-visualizing-fema-nri-data}{%
\section{\texorpdfstring{\textbf{HW \#2: Visualizing FEMA NRI
Data}}{HW \#2: Visualizing FEMA NRI Data}}\label{hw-2-visualizing-fema-nri-data}}

\hypertarget{learning-outcomes}{%
\subsection{\texorpdfstring{Learning
\textbf{Outcomes}}{Learning Outcomes}}\label{learning-outcomes}}

\begin{itemize}
\item
  identify which types of visualizations are most appropriate for your
  data and your audience
\item
  prepare (e.g.~clean, explore, wrangle) data so that it's appropriately
  formatted for building data visualizations
\item
  build effective, responsible, accessible, and aesthetically-pleasing,
  visualizations using the R programming language, and specifically
  \texttt{\{ggplot2\}} + ggplot2 extension packages
\end{itemize}

\hypertarget{description}{%
\subsection{\texorpdfstring{\textbf{Description}}{Description}}\label{description}}

\textbf{In class, we've been discussing strategies and considerations
for choosing the right graphic form to represent your data and convey
your intended message. Here, you'll apply what we're learning while
exploring natural hazards data, courtesy of the
\href{https://hazards.fema.gov/nri/}{FEMA Resilience Analysis and
Planning Tool (RAPT)}}.

\hypertarget{iv.-build-your-visualization}{%
\subsubsection{\texorpdfstring{\textbf{IV. Build your
visualization}}{IV. Build your visualization}}\label{iv.-build-your-visualization}}

Create a data viz that helps to answer the question, How do FEMA
National Risk Index scores for counties in California compare to those
in other states? This may require some data wrangling.

Your final visualization should:

\begin{itemize}
\item
  include data for the 50 US states only (no territories)
\item
  include a title (short, descriptive) \& subtitle (describes main
  takeaway) (see
  \href{https://clauswilke.com/dataviz/figure-titles-captions.html}{Fundamentals
  of Data Visualization, Ch 22} for an example), a caption (describes
  the data source, e.g.~``Data: FEMA National Risk Index (2025
  Release)''), and alt text (following the formula you recently
  practiced in
  \href{https://eds-240-data-viz.github.io/course-materials/lab/LAB-SLIDES-alt-text.html\#alt-text-formula}{lab};
  use the \texttt{fig-alt} code chunk option to apply your alt text)
\item
  consider and implement strategies for highlighting trends / important
  information (e.g.~arranging data, highlighting data, adjusting scales)
\item
  use custom colors (if applicable), rather than ggplot defaults
\item
  have an updated / polished theme
\end{itemize}

***You can adjust the aspect ratio of your plot (i.e.~its width to
height ratio) so that your data / groups are easy to read. Add the
\texttt{fig-asp} option to your code chunk YAML, which makes adjusting
aspect ratios for rendered outputs quite easy. Values \textgreater{} 1
make your plot taller and values \textless{} 1 make your plot wider.

\begin{Shaded}
\begin{Highlighting}[]
\CommentTok{\# Read in data }
\FunctionTok{library}\NormalTok{(tidyverse)}
\FunctionTok{library}\NormalTok{(here)}

\NormalTok{fema }\OtherTok{\textless{}{-}} \FunctionTok{read\_csv}\NormalTok{(}\FunctionTok{here}\NormalTok{(}\StringTok{"data"}\NormalTok{, }\StringTok{"National\_Risk\_Index\_Counties\_807384124455672111.csv"}\NormalTok{))}

\CommentTok{\# Select for the relevant variables and exclude non{-}US state designations }
\NormalTok{fema\_clean }\OtherTok{\textless{}{-}}\NormalTok{ fema }\SpecialCharTok{\%\textgreater{}\%} 
  \FunctionTok{filter}\NormalTok{(}\SpecialCharTok{!}\StringTok{\textasciigrave{}}\AttributeTok{State Name}\StringTok{\textasciigrave{}} \SpecialCharTok{\%in\%} \FunctionTok{c}\NormalTok{(}\StringTok{"District of Columbia"}\NormalTok{,}\StringTok{"American Samoa"}\NormalTok{, }\StringTok{"Guam"}\NormalTok{, }\StringTok{"Northern Mariana Islands"}\NormalTok{, }\StringTok{"Puerto Rico"}\NormalTok{, }\StringTok{"Virgin Islands"}\NormalTok{ )) }\SpecialCharTok{\%\textgreater{}\%} 
  \FunctionTok{select}\NormalTok{(}\AttributeTok{state =} \StringTok{"State Name Abbreviation"}\NormalTok{, }\AttributeTok{state\_name =} \StringTok{"State Name"}\NormalTok{, }\AttributeTok{county =} \StringTok{"County Name"}\NormalTok{, }\AttributeTok{nri =} \StringTok{"National Risk Index {-} Score {-} Composite"}\NormalTok{)}

\CommentTok{\# Observe the structure of the cleaned data}
\FunctionTok{glimpse}\NormalTok{(fema\_clean)}
\end{Highlighting}
\end{Shaded}

\begin{verbatim}
Rows: 3,143
Columns: 4
$ state      <chr> "AL", "AL", "AL", "AL", "AL", "AL", "AL", "AL", "AL", "AL",~
$ state_name <chr> "Alabama", "Alabama", "Alabama", "Alabama", "Alabama", "Ala~
$ county     <chr> "Autauga", "Baldwin", "Barbour", "Bibb", "Blount", "Bullock~
$ nri        <dbl> 57.56997, 96.72392, 48.12341, 39.12214, 68.47964, 25.25445,~
\end{verbatim}

\begin{Shaded}
\begin{Highlighting}[]
\CommentTok{\# Create a ranked bar plot with the averaged composite NRI score per state}
\NormalTok{fema\_clean }\SpecialCharTok{\%\textgreater{}\%}
  \FunctionTok{group\_by}\NormalTok{(state\_name) }\SpecialCharTok{\%\textgreater{}\%}
  
  \CommentTok{\# Calculate the average county NRI score by state}
  \FunctionTok{summarize}\NormalTok{(}\AttributeTok{mean\_nri =} \FunctionTok{mean}\NormalTok{(nri, }\AttributeTok{na.rm =}\NormalTok{ T)) }\SpecialCharTok{\%\textgreater{}\%} 
  
  \CommentTok{\# Create a new variable to group states }
  \FunctionTok{mutate}\NormalTok{(}
    \CommentTok{\# Reorder states by mean NRI }
    \AttributeTok{state =} \FunctionTok{fct\_reorder}\NormalTok{(state\_name, mean\_nri),}
    \CommentTok{\# Label California as its own category, then group all remaining states together }
    \AttributeTok{state\_group =} \FunctionTok{if\_else}\NormalTok{(state\_name }\SpecialCharTok{==} \StringTok{"California"}\NormalTok{, }\StringTok{"California"}\NormalTok{, }\StringTok{"Other States"}\NormalTok{)}
\NormalTok{  ) }\SpecialCharTok{\%\textgreater{}\%}  
 
  \CommentTok{\# Set the aesthetics for the ranked bar plot}
  \FunctionTok{ggplot}\NormalTok{(}\FunctionTok{aes}\NormalTok{(}\AttributeTok{x =}\NormalTok{ mean\_nri, }\AttributeTok{y =}\NormalTok{ state, }\AttributeTok{fill =}\NormalTok{ state\_group)) }\SpecialCharTok{+} \CommentTok{\# Set aesthetics}
  
  \FunctionTok{geom\_col}\NormalTok{() }\SpecialCharTok{+}
  
  \CommentTok{\# Label each bar with its corresponding average NRI score}
  \FunctionTok{geom\_text}\NormalTok{( }
    \FunctionTok{aes}\NormalTok{( }
      \AttributeTok{label =} \FunctionTok{round}\NormalTok{(mean\_nri, }\DecValTok{2}\NormalTok{)), }\CommentTok{\# Round each score to the nearest hundreth}
      \AttributeTok{hjust =} \SpecialCharTok{{-}}\FloatTok{0.1}\NormalTok{, }\CommentTok{\# Adjust the placement to scale}
      \AttributeTok{size =} \FloatTok{2.5} \CommentTok{\# Set a size for each labels}
\NormalTok{  ) }\SpecialCharTok{+}  
  
  \CommentTok{\# Rescale and adjust the positioning of labels on the x{-}axis}
  \FunctionTok{scale\_x\_continuous}\NormalTok{(}
    \AttributeTok{expand =} \FunctionTok{expansion}\NormalTok{(}\AttributeTok{mult =} \FunctionTok{c}\NormalTok{(}\DecValTok{0}\NormalTok{, }\FloatTok{0.05}\NormalTok{)), }\CommentTok{\# Remove gap at 0 }
     \AttributeTok{breaks =} \FunctionTok{seq}\NormalTok{(}\DecValTok{0}\NormalTok{, }\DecValTok{100}\NormalTok{, }\AttributeTok{by =} \DecValTok{10}\NormalTok{)) }\SpecialCharTok{+}     \CommentTok{\# Break every 10\% }
  
  \CommentTok{\# Assign colors to each state group (California vs Other States)}
    \FunctionTok{scale\_fill\_manual}\NormalTok{(}
    \AttributeTok{name =} \StringTok{""}\NormalTok{, }\CommentTok{\# Leave the legend title as an empty string}
    \AttributeTok{values =} \FunctionTok{c}\NormalTok{(}
      \StringTok{"California"} \OtherTok{=} \StringTok{"darkred"}\NormalTok{, }\CommentTok{\# Assign "California" a dark red color}
      \StringTok{"Other States"} \OtherTok{=} \StringTok{"grey60"} \CommentTok{\# Assign the grouped states besides California a grey color for contrast}
\NormalTok{    )) }\SpecialCharTok{+} 
      
  \CommentTok{\# Update the labels of the plot    }
  \FunctionTok{labs}\NormalTok{(}
    \AttributeTok{title =} \StringTok{"Average County National Risk Index (Composite Score) by State"}\NormalTok{, }\CommentTok{\# Add a title}
    \AttributeTok{subtitle =} \StringTok{"Each value represents the mean county{-}level NRI score for a state, with }\SpecialCharTok{\textbackslash{}n}\StringTok{California highlighted as the baseline for comparison."}\NormalTok{, }\CommentTok{\# Add a subtitle with a line break}
    \AttributeTok{x =} \StringTok{"Average Composite County National Risk Index (\%)"}\NormalTok{, }\CommentTok{\# Add an x{-}label}
    \AttributeTok{y =} \StringTok{"State"}\NormalTok{, }\CommentTok{\# Add a y{-}label}
    \AttributeTok{caption =} \StringTok{"Data Source: 2025 Release of FEMA National Risk Index (NRI). Analysis contains only the 50 US States."}\NormalTok{, }\CommentTok{\# Add a caption}
    \AttributeTok{alt =} \StringTok{"Horizontal bar chart showing average county National Risk Index (NRI) by state, with California highlighted in dark red and other states in grey, bars ordered from highest to lowest mean NRI, each bar labeled with its numeric mean NRI Score, x{-}axis showing NRI (\%) with breaks every 10\%,and  y{-}axis listing the states."}\NormalTok{) }\SpecialCharTok{+} 
 
  \CommentTok{\# Select a theme with no background grid    }
  \FunctionTok{theme\_classic}\NormalTok{() }\SpecialCharTok{+} 
      
  \CommentTok{\# Adjust the legend and caption placement    }
  \FunctionTok{theme}\NormalTok{(}\AttributeTok{legend.position =} \StringTok{"bottom"}\NormalTok{, }\CommentTok{\# Move the legend position to bottom of plot}
        \AttributeTok{plot.caption.position =} \StringTok{"plot"}\NormalTok{, }\CommentTok{\# Align the caption with plot}
        \AttributeTok{plot.caption =} \FunctionTok{element\_text}\NormalTok{(}\AttributeTok{hjust =} \DecValTok{0}\NormalTok{)) }\CommentTok{\# Adjust the position horizontal to the plot}
\end{Highlighting}
\end{Shaded}

\begin{figure}[H]

{\centering \includegraphics{eds-240-hw-2_files/figure-pdf/unnamed-chunk-2-1.pdf}

}

\end{figure}

\hypertarget{v.-answer-some-questions}{%
\subsubsection{\texorpdfstring{\textbf{V. Answer some
questions}}{V. Answer some questions}}\label{v.-answer-some-questions}}

\begin{itemize}
\item
  \textbf{1.} What are your variables of interest and what kinds of data
  (e.g.~numeric, categorical, ordered, etc.) are they (a bullet point
  list is fine)?

  \begin{itemize}
  \tightlist
  \item
  \end{itemize}
\item
  \textbf{2.} How did you decide which type of graphic form was best
  suited for answering the question? What alternative graphic forms
  could you have used instead? Why did you settle on this particular
  graphic form?

  \begin{itemize}
  \tightlist
  \item
    ``Because FEMA's NRI is a standardized composite index representing
    relative county-level risk, averaging county scores within states
    provides a reasonable summary of typical risk levels for
    state-to-state comparison.''
  \end{itemize}
\item
  \textbf{3.} Summarize your main finding in no more than two sentences.
\item
  \textbf{4.} What modifications did you make to this visualization to
  make it more easily readable?
\item
  \textbf{5.} Is there anything you wanted to implement, but didn't know
  how? If so, please describe.
\end{itemize}

\hypertarget{rubric-specifications}{%
\subsection{\texorpdfstring{\textbf{Rubric
(specifications)}}{Rubric (specifications)}}\label{rubric-specifications}}

To receive a ``Satisfactory'' on HW \#2:

\begin{itemize}
\item
  \textbf{Create one visualization} that answers the question,
  \emph{``How do FEMA National Risk Index scores for counties in
  California compare to those in other states?''} and fulfills the
  requirements listed in Part IV.
\item
  \textbf{All code should follow appropriate styling conventions and
  include clear, informative annotations.} We will not enforce any
  particular code style guide (you may use conventions from prior
  courses if you prefer; we recommend following
  \href{https://eds-240-data-viz.github.io/clean-code-guide.html}{these
  conventions} from the
  \href{https://style.tidyverse.org/index.html}{Tidyverse style guide}
  if you're unsure where to start). You don't need to explain every
  line, but your annotations should make it easy to follow your thinking
  and understand important decisions in your code.
\item
  \textbf{Answer all five Part V questions.} There are no strict length
  requirements for free-response questions, however we expect that you
  answer them thoughtfully and fully.
\item
  \textbf{Ensure that everything described above is included in your
  neatly-organized and polished \texttt{HW2.qmd} file} -- this includes
  appropriate document YAML (at a minimum, title, author, date), clearly
  labeled sections, appropriately-formatted prose, appropriate code
  chunk options (e.g.~code should render and execute, but warnings and
  messages should be suppressed, long data frames should not be printed
  out, etc.)
\item
  \textbf{All your work is housed in an organized and polished
  \texttt{eds240-nri-acs-viz} GitHub repository.} Your README includes
  all required information, as described in the
  \href{https://ucsb-meds.github.io/README-guidelines/}{MEDS README
  Guidelines}. Your data folder should is added to your
  \texttt{.gitignore}.
\item
  \textbf{Render \texttt{HW2.qmd} as a PDF.} Double check that all of
  your outputs, code, and prose are visible.
\item
  \textbf{Upload your PDF to Gradescope} by 11:59pm on Wed 01/28/2026.
\item
  \textbf{Add your GitHub repo URL to this
  \href{https://docs.google.com/spreadsheets/d/1fERw6kvSyAf30rY9ZHtFRIu01wM3qGa8cBQU-AGzh-g/edit?usp=sharing}{Google
  Sheet}} by 11:59pm on Wed 01/28/2026 (so that we can find your README)
\item
  \textbf{Submit a \href{https://forms.gle/uCw7U5jzJAKahjA4A}{Generative
  AI Statement of Use}} for HW \#2 by 11:59pm on Wed 01/28/2026.
\end{itemize}



\end{document}
